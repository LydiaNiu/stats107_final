% Options for packages loaded elsewhere
\PassOptionsToPackage{unicode}{hyperref}
\PassOptionsToPackage{hyphens}{url}
%
\documentclass[
]{article}
\usepackage{amsmath,amssymb}
\usepackage{iftex}
\ifPDFTeX
  \usepackage[T1]{fontenc}
  \usepackage[utf8]{inputenc}
  \usepackage{textcomp} % provide euro and other symbols
\else % if luatex or xetex
  \usepackage{unicode-math} % this also loads fontspec
  \defaultfontfeatures{Scale=MatchLowercase}
  \defaultfontfeatures[\rmfamily]{Ligatures=TeX,Scale=1}
\fi
\usepackage{lmodern}
\ifPDFTeX\else
  % xetex/luatex font selection
\fi
% Use upquote if available, for straight quotes in verbatim environments
\IfFileExists{upquote.sty}{\usepackage{upquote}}{}
\IfFileExists{microtype.sty}{% use microtype if available
  \usepackage[]{microtype}
  \UseMicrotypeSet[protrusion]{basicmath} % disable protrusion for tt fonts
}{}
\makeatletter
\@ifundefined{KOMAClassName}{% if non-KOMA class
  \IfFileExists{parskip.sty}{%
    \usepackage{parskip}
  }{% else
    \setlength{\parindent}{0pt}
    \setlength{\parskip}{6pt plus 2pt minus 1pt}}
}{% if KOMA class
  \KOMAoptions{parskip=half}}
\makeatother
\usepackage{xcolor}
\usepackage[margin=1in]{geometry}
\usepackage{graphicx}
\makeatletter
\def\maxwidth{\ifdim\Gin@nat@width>\linewidth\linewidth\else\Gin@nat@width\fi}
\def\maxheight{\ifdim\Gin@nat@height>\textheight\textheight\else\Gin@nat@height\fi}
\makeatother
% Scale images if necessary, so that they will not overflow the page
% margins by default, and it is still possible to overwrite the defaults
% using explicit options in \includegraphics[width, height, ...]{}
\setkeys{Gin}{width=\maxwidth,height=\maxheight,keepaspectratio}
% Set default figure placement to htbp
\makeatletter
\def\fps@figure{htbp}
\makeatother
\setlength{\emergencystretch}{3em} % prevent overfull lines
\providecommand{\tightlist}{%
  \setlength{\itemsep}{0pt}\setlength{\parskip}{0pt}}
\setcounter{secnumdepth}{-\maxdimen} % remove section numbering
\ifLuaTeX
  \usepackage{selnolig}  % disable illegal ligatures
\fi
\usepackage{bookmark}
\IfFileExists{xurl.sty}{\usepackage{xurl}}{} % add URL line breaks if available
\urlstyle{same}
\hypersetup{
  pdftitle={FinalReport\_Niu\_Lydia.Rmd},
  hidelinks,
  pdfcreator={LaTeX via pandoc}}

\title{FinalReport\_Niu\_Lydia.Rmd}
\author{}
\date{\vspace{-2.5em}2025-03-15}

\begin{document}
\maketitle

\section{I am the title}\label{i-am-the-title}

\subsection{ABSTRACT}\label{abstract}

\subsection{INTRODUCTION}\label{introduction}

\subsection{DATA}\label{data}

\subsubsection{variables:}\label{variables}

\begin{itemize}
\tightlist
\item
  Sleep Disorder: The presence or absence of a sleep disorder in the
  person (None, Insomnia, Sleep Apnea).
\item
  Quality of Sleep (Scale: 1-10): A subjective rating of the quality of
  sleep, ranging from 1 to 10.
\item
  Stress Level (Scale: 1-10): A subjective rating of the stress level
  experienced by the person, ranging from 1 to 10.
\end{itemize}

\subsection{VISUALIZATION}\label{visualization}

This project uses a \textbf{t-test} because it is a statistical method
specifically designed to \textbf{compare the means of two groups} and
determine whether the observed differences are statistically
significant. Since our goal is to analyze whether \textbf{sleep quality
and stress levels differ between individuals with and without sleep
disorders (None, Sleep Apnea, Insomnia)}, the t-test is an appropriate
choice. It allows us to test the null hypothesis that the means of two
groups are equal, while accounting for variability in the data. By using
the \textbf{p-value and confidence intervals}, we can assess whether the
differences we observe are due to chance or represent a meaningful
pattern in sleep health.

\subsubsection{Density plots for 6
tests:}\label{density-plots-for-6-tests}

\newline

\includegraphics{FinalReport_Niu_Lydia_files/figure-latex/density_plots_page1-1.pdf}

\includegraphics{FinalReport_Niu_Lydia_files/figure-latex/density_plots_page2-1.pdf}

\subsubsection{Box plots for the 6
testsd}\label{box-plots-for-the-6-testsd}

\includegraphics{FinalReport_Niu_Lydia_files/figure-latex/density_plots_page3-1.pdf}

\includegraphics{FinalReport_Niu_Lydia_files/figure-latex/density_plots_page4-1.pdf}

\subsubsection{Four-quadrant distribution
chart}\label{four-quadrant-distribution-chart}

\begin{verbatim}
## If reinstallation fails, try install_tinytex() again. Then install the following packages:
## 
## tinytex::tlmgr_install(c("amscls", "amsfonts", "amsmath", "atbegshi", "atveryend", "auxhook", "babel", "bibtex", "bigintcalc", "bitset", "bookmark", "booktabs", "cm", "ctablestack", "dehyph", "dvipdfmx", "dvips", "ec", "epstopdf", "epstopdf-pkg", "etex", "etexcmds", "etoolbox", "euenc", "extractbb", "fancyvrb", "filehook", "firstaid", "float", "fontspec", "framed", "geometry", "gettitlestring", "glyphlist", "graphics", "graphics-cfg", "graphics-def", "helvetic", "hycolor", "hyperref", "hyph-utf8", "hyphen-base", "iftex", "inconsolata", "infwarerr", "intcalc", "knuth-lib", "kpathsea", "kvdefinekeys", "kvoptions", "kvsetkeys", "l3backend", "l3kernel", "l3packages", "latex", "latex-amsmath-dev", "latex-bin", "latex-fonts", "latex-tools-dev", "latexconfig", "latexmk", "letltxmacro", "lm", "lm-math", "ltxcmds", "lua-alt-getopt", "lua-uni-algos", "luahbtex", "lualatex-math", "lualibs", "luaotfload", "luatex", "luatexbase", "mdwtools", "metafont", "mfware", "modes", "natbib", "pdfescape", "pdftex", "pdftexcmds", "plain", "psnfss", "refcount", "rerunfilecheck", "scheme-infraonly", "selnolig", "stringenc", "symbol", "tex", "tex-ini-files", "texlive-scripts", "texlive-scripts-extra", "texlive.infra", "times", "tipa", "tlgpg", "tools", "unicode-data", "unicode-math", "uniquecounter", "url", "xcolor", "xetex", "xetexconfig", "xkeyval", "xunicode", "zapfding"))
\end{verbatim}

\begin{verbatim}
## The directory /Users/user/Library/TinyTeX/texmf-local is not empty. It will be backed up to /var/folders/_2/3lrnsn7x5615sf09xw7cfcfm0000gn/T//RtmpGeK28v/file11d647e927820 and restored later.
\end{verbatim}

\begin{verbatim}
## tlmgr install tlgpg
\end{verbatim}

\begin{verbatim}
## tlmgr update --self
\end{verbatim}

\begin{verbatim}
## tlmgr install tlgpg
\end{verbatim}

\begin{verbatim}
## tlmgr --repository http://www.preining.info/tlgpg/ install tlgpg
\end{verbatim}

\begin{verbatim}
## tlmgr option repository 'https://ctan.math.illinois.edu/systems/texlive/tlnet'
\end{verbatim}

\begin{verbatim}
## tlmgr update --list
\end{verbatim}

\subsection{ANALYSIS}\label{analysis}

\section{Analyzing the Four-quadrant Distribution
Chart}\label{analyzing-the-four-quadrant-distribution-chart}

Clustering patterns: Sleep Apnea (square) tends to be in the upper left
quadrant, meaning higher sleep quality and lower stress. Insomnia (solid
circle) tends to be in the lower right quadrant, meaning higher stress
and lower sleep quality. Normal individuals (triangle) are scattered but
mostly near the center.

\subsection{CONCLUSION}\label{conclusion}

\end{document}
